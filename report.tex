\documentclass{article}
\usepackage{xspace,graphicx,amsmath,amsthm,amssymb,xcolor}
\setlength{\topmargin}{0.1in}
\setlength{\oddsidemargin}{0in}
\setlength{\evensidemargin}{0in}
\setlength{\headheight}{0in}
\setlength{\headsep}{0in}
\setlength{\textheight}{9in}
\setlength{\textwidth}{6.5in}
\title{CS531 Programming assignment \#1}
\author{Matt Meyn, Durga Harish, and Jon Dodge}


\begin{document}
\maketitle

\section{Introduction}
The goal of this project is to assess the results of various agent control strategies in a simple path planning domain.
The three strategies we explore are
\begin{enumerate}
\item Memoryless deterministic reflex agent
\item Randomized reflex agent
\item Deterministic Model-based reflex agent
\end{enumerate}

We found... FIXME provide summary of results once that is known

\section{Description of agents}
Here we will describe our agents as a set of rules, and give a diagram illustrating the path it takes in each environment.

\subsection{Memoryless deterministic reflex agent}
Essentially, this agent will move forward until it hits a wall, then turn right.

FIXME typeset this pseudocode better

		if isSpaceDirty:
			return self.ActSuckDirt
		
		if not isFacingWall:
			return self.ActMove
			
		if isHome:
			return self.ActTurnOff
		else:
			return self.ActTurnRight

FIXME generate diagrams

\subsection{Randomized reflex agent}
Essentially, this agent will randomly choose actions with a uniform distribution between the available actions. 
In general, available actions are to move, turn left, and turn right, though the existence of a wall may remove the availability of the move option.
The case that the agent is on the home square is treated similarly, it has equal probability to turn off or try to continue vacuuming.

FIXME typeset this pseudocode better

	if isSpaceDirty:
		return self.ActSuckDirt

	if isHome:
		if random.randint(0, 5) == 0:
			return self.ActTurnOff

	if isFacingWall:
		return self.TurnRandom()

	else:
		if random.randint(0, 3) == 0:
			return self.TurnRandom()
		else:
			return self.ActMove

FIXME generate diagrams


\subsection{Deterministic Model-based reflex agent}
FIXME implement this
our first plan is to do a spiraling space filling curve, and our second plan is to do a snaking space filling curve


FIXME typeset this pseudocode better



FIXME generate diagrams





\section{Experimental Setup}
We have conducted experiments in 

\section{Results}

\section{Discussion}
\begin{enumerate}
\item What is the best possible performance achievable by the simple reflex agent in the two environments? 
What prevents it from achieving the goal of cleaning the rooom perfectly in each case? 


\item How well does the random agent perform? 
Tune any parameters of the random agent to improve its performance. 
Give a table showing the number of actions it took to clean 90\% of the room for each trial. 
What is the average of these numbers for the best 45 trials? 
What are the costs and benefits of randomness in agent design? 

\item How does the memory-based deterministic agent perform in the two environments?
Is it able to clean the room perfectly? 
How long does it take? 
Can the agent be improved with more memory?

\item What are the tradeoffs between the random and determistic agents? 
How would you design better agents for more complex environments, say with polygonal obstacles? 

\item What did you learn from this experiment? 
Were you surprised by anything? 

\end{enumerate}
\end{document}